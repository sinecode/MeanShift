\documentclass{beamer}

\usepackage[utf8]{inputenc}
\usepackage{algorithm}
\usepackage{algpseudocode}

\usetheme[%
    numbering=fraction,
    progressbar=foot,
]{metropolis}

\setbeamercolor{frametitle}{use=normal text}


\title{Parallelization of the Mean Shift Clustering with OpenMP}
\author{Emilio Cecchini}
\institute{
    Università degli Studi di Firenze \\
    \medskip
    \textit{emilio.cecchini@stud.unfi.it}
}
\date{\today}


\begin{document}


\maketitle


\begin{frame}{Overview}
\tableofcontents
\end{frame}


\section{The Mean Shift Clustering}

\begin{frame}{Mean Shift key concepts}

\begin{itemize}
\item
Non-parametric technique to find the maxima of a density function.
\item
At each step, a \textit{kernel function} is applied to each point that causes the points to shift in the direction of the local maxima determined by the kernel.
\end{itemize}

\end{frame}


\begin{frame}{Gaussian Kernel}

\begin{itemize}
\item
There are many different types of kernel, the most used is the \textit{Gaussian kernel}:

\begin{align*}
k(x) =  e^{-\dfrac{x}{2\sigma^2}}
\end{align*}
\item
The standard deviation $\sigma$ is the bandwidth parameter, with a high bandwith value you will get a few large clusters and vice versa.
\end{itemize}

\end{frame}


\begin{frame}{Mean Shift Clustering}
Suppose $x$ is a point to be shifted and $N(x)$ are the sets of points near to that point. Let $dist(x, x_i)$ be the distance from the point $x$ to the point $x_i$. The new position $x'$ where $x$ has to be shifted is computed as follows:

\begin{align*}
x' = \dfrac{\sum_{x_i \in N(x)} k(dist(x,x_i)^2) x_i}{\sum_{x_i \in N(x)} k(dist(x, x_i)^2)}
\end{align*}

The mean shift algorithm applies that formula to each point iteratively until they converge, that is until the position does not change.
\end{frame}


\section{Sequential implementation}


\begin{frame}{Sequential implementation}

\begin{algorithm}[H]
\caption{Mean shift core}
\begin{algorithmic}[1]

    \While{allPointsHaveStoppedShifting()}
            \For{each point $p$}
                \If{hasStoppedShifting($p$)}
                    \State \textbf{continue}
                \EndIf
            \State shift($p$)
            \EndFor
    \EndWhile

\end{algorithmic}
\end{algorithm}

\end{frame}


\section{OpenMP}


\begin{frame}{First slide}
\end{frame}


\begin{frame}{Second slide}
\end{frame}


\section{Parallelization of the Mean Shift with OpenMP}


\begin{frame}{First slide}
\end{frame}


\begin{frame}{Second slide}
\end{frame}


\section{Speedup}


\begin{frame}{First slide}
\end{frame}


\begin{frame}{Second slide}
\end{frame}


\end{document}
